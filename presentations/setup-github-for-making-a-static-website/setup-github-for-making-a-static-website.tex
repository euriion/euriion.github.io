% Options for packages loaded elsewhere
\PassOptionsToPackage{unicode}{hyperref}
\PassOptionsToPackage{hyphens}{url}
%
\documentclass[
]{article}
\usepackage{lmodern}
\usepackage{amssymb,amsmath}
\usepackage{ifxetex,ifluatex}
\ifnum 0\ifxetex 1\fi\ifluatex 1\fi=0 % if pdftex
  \usepackage[T1]{fontenc}
  \usepackage[utf8]{inputenc}
  \usepackage{textcomp} % provide euro and other symbols
\else % if luatex or xetex
  \usepackage{unicode-math}
  \defaultfontfeatures{Scale=MatchLowercase}
  \defaultfontfeatures[\rmfamily]{Ligatures=TeX,Scale=1}
\fi
% Use upquote if available, for straight quotes in verbatim environments
\IfFileExists{upquote.sty}{\usepackage{upquote}}{}
\IfFileExists{microtype.sty}{% use microtype if available
  \usepackage[]{microtype}
  \UseMicrotypeSet[protrusion]{basicmath} % disable protrusion for tt fonts
}{}
\makeatletter
\@ifundefined{KOMAClassName}{% if non-KOMA class
  \IfFileExists{parskip.sty}{%
    \usepackage{parskip}
  }{% else
    \setlength{\parindent}{0pt}
    \setlength{\parskip}{6pt plus 2pt minus 1pt}}
}{% if KOMA class
  \KOMAoptions{parskip=half}}
\makeatother
\usepackage{xcolor}
\IfFileExists{xurl.sty}{\usepackage{xurl}}{} % add URL line breaks if available
\IfFileExists{bookmark.sty}{\usepackage{bookmark}}{\usepackage{hyperref}}
\hypersetup{
  pdftitle={Github에 도메인 연결해서 고정 웹페이지 서비스하기},
  pdfauthor={홍성학},
  hidelinks,
  pdfcreator={LaTeX via pandoc}}
\urlstyle{same} % disable monospaced font for URLs
\usepackage[margin=1in]{geometry}
\usepackage{graphicx,grffile}
\makeatletter
\def\maxwidth{\ifdim\Gin@nat@width>\linewidth\linewidth\else\Gin@nat@width\fi}
\def\maxheight{\ifdim\Gin@nat@height>\textheight\textheight\else\Gin@nat@height\fi}
\makeatother
% Scale images if necessary, so that they will not overflow the page
% margins by default, and it is still possible to overwrite the defaults
% using explicit options in \includegraphics[width, height, ...]{}
\setkeys{Gin}{width=\maxwidth,height=\maxheight,keepaspectratio}
% Set default figure placement to htbp
\makeatletter
\def\fps@figure{htbp}
\makeatother
\setlength{\emergencystretch}{3em} % prevent overfull lines
\providecommand{\tightlist}{%
  \setlength{\itemsep}{0pt}\setlength{\parskip}{0pt}}
\setcounter{secnumdepth}{-\maxdimen} % remove section numbering

\title{Github에 도메인 연결해서 고정 웹페이지 서비스하기}
\author{홍성학}
\date{2021년 7월 2일}

\begin{document}
\maketitle

\hypertarget{uxb0b4uxc6a9}{%
\section{내용}\label{uxb0b4uxc6a9}}

\begin{itemize}
\tightlist
\item
  배경설명
\item
  준비물
\item
  도메인 설정하기
\item
  Github 설정하기
\item
  확인하기
\end{itemize}

\hypertarget{uxbc30uxacbd-uxc124uxba85}{%
\section{배경 설명}\label{uxbc30uxacbd-uxc124uxba85}}

\begin{itemize}
\tightlist
\item
  R로 작성한 문서를 HTML로 만들고 서비스를 하려면 서버 또는 VPS 호스팅이
  필요함
\item
  호스팅 서비스를 이용하는데 비용을 지불해야함
\item
  호스팅 자원을 관리하는데 시간과 노력도 필요함
\item
  비용과 노력을 최소화해서 정적인 HTML을 서비스할 수 있는 방법을
  Github이 제공하고 있음
\item
  과거 10여년 동안 많은 과학자, 데이터 분석가, 개발자들이 이 방법을
  이용해 왔음
\item
  아직까지 가장 저렴하면서 쉬운 방법은 Github + Domain 설정 변경
\end{itemize}

\hypertarget{uxc900uxbe44uxbb3c}{%
\section{준비물}\label{uxc900uxbe44uxbb3c}}

\textbf{도메인}

\begin{itemize}
\tightlist
\item
  구매하거나 이미 소유하고 있는 것을 사용
\item
  구매 비용은 .com도메인은 1년에 13,500원 (가비아 기준)
\end{itemize}

\textbf{Github 레파지토리}

\begin{itemize}
\tightlist
\item
  Github 계정이 없으면 하나 새로 생성해야 함
\item
  도메인을 연결할 Public 레파지토리를 생성
\item
  예전에는 ``tidyverse-kr.github.io'' 와 같은 정해진 이름 규칙이
  있었지만 지금은 아님
\item
  Private은 도메인을 연결할 수 없음
\end{itemize}

\hypertarget{uxac04uxb7b5uxd55c-uxc21cuxc11c-uxc694uxc57d}{%
\section{간략한 순서
요약}\label{uxac04uxb7b5uxd55c-uxc21cuxc11c-uxc694uxc57d}}

\begin{enumerate}
\def\labelenumi{\arabic{enumi}.}
\tightlist
\item
  도메인 레코드 설정하기
\item
  Github 레파지토리 설정하기
\item
  초기 HTML 파일 업로드하기
\item
  확인하기
\end{enumerate}

\hypertarget{uxb3c4uxba54uxc778-uxc124uxc815-uxd558uxae30---uxc694uxc57d}{%
\section{도메인 설정 하기 -
요약}\label{uxb3c4uxba54uxc778-uxc124uxc815-uxd558uxae30---uxc694uxc57d}}

\begin{itemize}
\tightlist
\item
  도메인에 CNAME 레코드를 추가 또는 변경하기
\item
  소유한 도메인의 레지스타별로 DNS 레코드 수정하는 화면과 방법이 다름
\item
  각 도메인의 레지스타의 도움말 또는 문서를 참고해야 함
\end{itemize}

\hypertarget{uxb3c4uxba54uxc778-uxc124uxc815-1}{%
\section{도메인 설정 1}\label{uxb3c4uxba54uxc778-uxc124uxc815-1}}

\begin{figure}
\centering
\includegraphics[width=6.25in,height=\textheight]{./domain-01.png}
\caption{setting-dns-record-01}
\end{figure}

\hypertarget{uxb3c4uxba54uxc778-uxc124uxc815-2}{%
\section{도메인 설정 2}\label{uxb3c4uxba54uxc778-uxc124uxc815-2}}

\begin{itemize}
\tightlist
\item
  설정을 하려는 도메인을 찾아서 DNS 레코드 설정으로 들어갈 것
\item
  진입 방법은 레지스트라(registrar)별로 다름
  \includegraphics[width=6.25in,height=\textheight]{./domain-02.png}
\end{itemize}

\hypertarget{uxb3c4uxba54uxc778-uxc124uxc815-3}{%
\section{도메인 설정 3}\label{uxb3c4uxba54uxc778-uxc124uxc815-3}}

\begin{itemize}
\tightlist
\item
  아래의 화면은 이미 등록된 상태를 보여주는 것
\item
  처음 하는 것이면 CNAME이 등록되어 있지 않을 것이므로 아래의 화면과
  다름
\end{itemize}

\begin{figure}
\centering
\includegraphics[width=6.25in,height=\textheight]{./domain-03.png}
\caption{setting-dns-record-03}
\end{figure}

\hypertarget{uxb3c4uxba54uxc778-uxc124uxc815-4}{%
\section{도메인 설정 4}\label{uxb3c4uxba54uxc778-uxc124uxc815-4}}

\begin{itemize}
\tightlist
\item
  CNAME에 입력할 도메인 주소는 ``계정명.github.io.''
\item
  끝에 ``.''이 붙으므로 주의
\item
  이 예에서는 ``tidyverse-kr''이 github 계정명
\end{itemize}

\begin{figure}
\centering
\includegraphics[width=6.25in,height=\textheight]{./domain-04.png}
\caption{setting-dns-record-04}
\end{figure}

\hypertarget{githubuxc5d0uxc11c-uxd398uxc774uxc9c0uxc124uxc815-1}{%
\section{github에서 페이지설정
1}\label{githubuxc5d0uxc11c-uxd398uxc774uxc9c0uxc124uxc815-1}}

\begin{figure}
\centering
\includegraphics[width=6.25in,height=\textheight]{./domain-01.png}
\caption{setting-dns-record-01}
\end{figure}

\hypertarget{githubuxc5d0uxc11c-uxd398uxc774uxc9c0uxc124uxc815-2}{%
\section{github에서 페이지설정
2}\label{githubuxc5d0uxc11c-uxd398uxc774uxc9c0uxc124uxc815-2}}

\begin{itemize}
\tightlist
\item
  웹루트로 사용할 브랜치와 디렉토리를 지정하면 CNAME이라는 파일이
  자동으로 생성되서 저장됨
\item
  ``Enable HTTPS''를 체크하면 ``\url{https://use-kr.com}'' 과 같이
  https로 서비스할 수 있음
\end{itemize}

\begin{figure}
\centering
\includegraphics[width=6.25in,height=\textheight]{./domain-02.png}
\caption{setting-dns-record-02}
\end{figure}

\hypertarget{uxd655uxc778uxd558uxae30}{%
\section{확인하기}\label{uxd655uxc778uxd558uxae30}}

\begin{itemize}
\tightlist
\item
  index.html을 비롯한 내용물을 github 레파지토리에 올림
\item
  웹브라우저로 접속해서 확인. 적용되는데 최대 5분까지 걸림
\item
  예시 \href{}{https://use-r.kr}
\end{itemize}

\hypertarget{uxae30uxd0c0-uxacf5uxc720}{%
\section{기타 공유}\label{uxae30uxd0c0-uxacf5uxc720}}

\begin{itemize}
\tightlist
\item
  Knitr와 Rmarkdown을 이용해서 연동하기에 매우 편리함
\item
  Blogdown, Bookdown,\ldots down 시리즈의 결과물을 저장하고 공유하기에
  매우 적합함
\item
  함께 연동할 수 있는 Jeykyl, Hyde, Hugo, Hexo 등 많은 다른 솔루션이
  있음
\end{itemize}

\hypertarget{uxb9c8uxce58uxba70}{%
\section{마치며}\label{uxb9c8uxce58uxba70}}

\begin{itemize}
\tightlist
\item
  이 슬라이드는 Rmarkdown revealjs 형식으로 만든 것입니다.
\item
  슬라이드 소스는
  \href{https://github.com/aidenhong/aidenhong.github.io/tree/master/presentations/setup-github-for-making-a-static-website}{여기}에
  있습니다.
\item
  자료는 마음대로 사용하셔도 됩니다.
\end{itemize}

\end{document}
